\documentclass{article}

\usepackage[top=3cm, bottom=3cm, left=3cm, right=3cm]{geometry}

\usepackage[utf8]{inputenc}
\usepackage[T1]{fontenc}
\usepackage[frenchb]{babel}

\usepackage[a4paper,colorlinks,linkcolor=darkgray,citecolor=red,urlcolor=blue]{hyperref}
\usepackage{graphicx}
\usepackage{caption}
\usepackage{subcaption}
\usepackage{amsthm}
\usepackage{listings}
\usepackage{tikz}
\usepackage{algorithm}
\usepackage{amsmath}
\usepackage[noend]{algpseudocode}
\usepackage{amsfonts}
\usepackage{amssymb}
\usepackage{authblk}

\renewcommand\Affilfont{\small}

\title{Projet Images}

\author[$\dag$]{Laureline Pinault et Raphaël Charrondière}
\affil[$\dag$]{ENS de Lyon, France}

\date{Lundi 4 mai 2015}

\begin{document}

  \maketitle

  \begin{abstract}
   
  \end{abstract}
  
  \newpage

  \tableofcontents

  \newpage
  
  \section{Introduction}
  
    \subsection{Remarques préliminaires}
    
    \subsection{Objectifs du projet}
  
    \subsection{Organisation du projet}
  
  \section{Traitement des images}
  
    \subsection{Traitement anti-inversion}
    
      \begin{figure}[!h]
	\centering
	\begin{subfigure}{.49\textwidth}
	  \begin{subfigure}{.49\textwidth}
	    \centering
	    \includegraphics[scale=0.25]{Illustrations/rat-8.png}
	    \label{1strat}
	  \end{subfigure}
	  \begin{subfigure}{.49\textwidth}
	    \centering
	    \includegraphics[scale=0.25]{Illustrations/rat-8-inverse.png}
	    \label{1strat-inverse}
	  \end{subfigure}
	  \subcaption{Ce rat n'a pas été inversé}
	\end{subfigure}
	\begin{subfigure}{.49\textwidth}
	  \begin{subfigure}{.49\textwidth}
	    \centering
	    \includegraphics[scale=0.25]{Illustrations/rat-9.png}
	  \label{2ndrat}
	  \end{subfigure}
	  \begin{subfigure}{.49\textwidth}
	    \centering
	    \includegraphics[scale=0.25]{Illustrations/rat-9-inverse.png}
	  \label{2ndrat-inverse}
	  \end{subfigure}
	  \subcaption{Ce rat a été inversé}
	\end{subfigure}
	\caption{Inversion de l'objet rat}
	\label{anti-inversion}
      \end{figure} 
  
    \subsection{Traitement anti-bruit}
    
    \subsection{Remplissage}
    
      \begin{figure}[!h]
	\centering
	\begin{subfigure}{.25\textwidth}
	  \centering
	  \includegraphics[scale=0.30]{Illustrations/pocket-20.png}
	  \label{pocket-non-rempli}
	\end{subfigure}
	\begin{subfigure}{.25\textwidth}
	  \centering
	  \includegraphics[scale=0.30]{Illustrations/pocket-20-rempli.png}
	\label{pocket-rempli}
	\end{subfigure}
	\caption{Remplissage de l'objet pocket}
	\label{remplissage}
      \end{figure}
      
      \begin{figure}[!h]
	\centering
	\begin{subfigure}{.47\textwidth}
	  \begin{subfigure}{.52\textwidth}
	    \centering
	    \includegraphics[scale=0.295]{Illustrations/butterfly-10-rempli.png}
	    \label{butterfly-rempli}
	  \end{subfigure}
	  \begin{subfigure}{.45\textwidth}
	    \centering
	    \includegraphics[scale=0.16]{Illustrations/device9-6-rempli.png}
	    \label{spirale-rempli}
	  \end{subfigure}
	  \subcaption{Certains détails ne sont pas remplis}
	\end{subfigure}
	\begin{subfigure}{.44\textwidth}
	  \begin{subfigure}{.46\textwidth}
	    \centering
	    \includegraphics[scale=0.25]{Illustrations/cup-7-rempli.png}
	  \label{1stcup-rempli}
	  \end{subfigure}
	  \begin{subfigure}{.46\textwidth}
	    \centering
	    \includegraphics[scale=0.275]{Illustrations/cup-12-rempli.png}
	  \label{2ndcup-rempli}
	  \end{subfigure}
	  \subcaption{Les tasses ne sont pas remplies pareil}
	\end{subfigure}
	\caption{Problèmes soulevés par le remplissage des objets}
	\label{problèmes-remplissage}
      \end{figure}      
   
  \section{Estimateurs}
  
    \subsection{Quelques estimateurs simples et complémentaires}

      \subsubsection{Solidité}
      
	\paragraph{Principe}
	
	  \begin{figure}[!h]
	    \centering
	    \begin{subfigure}{.25\textwidth}
	      \centering
	      \includegraphics[scale=0.30]{Illustrations/horse-20.png}
	      \label{horse}
	    \end{subfigure}
	    \begin{subfigure}{.25\textwidth}
	      \centering
	      \includegraphics[scale=0.30]{Illustrations/horse-20-rempli-convexhull.png}
	    \label{horse-rempli-convexhull}
	    \end{subfigure}
	    \caption{Solidité de l'objet horse}
	    \label{solidité}
	  \end{figure}
	
	\paragraph{Programmation}
      
	\paragraph{Résistance aux homothéties}
	
	  blabla \\
	
	  \begin{tabular}{|c|c|c|c|c|c|c|c|}
	    \hline
	    \multicolumn{2}{|c|}{Cheval} & \multicolumn{2}{|c|}{Pomme} & \multicolumn{2}{|c|}{Device} & \multicolumn{2}{|c|}{Lézard} \\
	    \multicolumn{2}{|c|}{\includegraphics[scale=0.15]{Illustrations/horse-20.png}} 
	    & \multicolumn{2}{|c|}{\includegraphics[scale=0.15]{Illustrations/apple-3.png}} 
	    & \multicolumn{2}{|c|}{\includegraphics[scale=0.075]{Illustrations/device7-1.png}} 
	    & \multicolumn{2}{|c|}{\includegraphics[scale=0.09]{Illustrations/lizzard-13.png}} \\
	    \hline
	    \textbf{Echelle} & \textbf{Solidité} & \textbf{Echelle} & \textbf{Solidité} & \textbf{Echelle} & \textbf{Solidité} & \textbf{Echelle} & \textbf{Solidité} \\
	    \hline
	    0.5 & 0.584583 & 0.5 & 0.933504 & 0.5 & 0.77016 & 0.5 & 0.767554 \\
	    \hline
	    1 & 0.577491 & 1 & 0.931911 & 1 & 0.768281 & 1 & 0.762166 \\
	    \hline
	    2 & 0.57244 & 2 & 0.930546 & 2 & 0.767288 & 2 & 0.75678 \\
	    \hline
	    5 & 0.569273 & 5 & 0.929267 & 5 & 0.703692 & 5 & 0.7533 \\
	    \hline
	  \end{tabular}
	  
	\paragraph{Résultats}

      \subsubsection{Excentricté}
      
      \subsubsection{Energie de flexion moyenne}
    
    \subsection{Transformée de Fourier}
    
    \subsection{}
  
  \section{Résultats}
  
  \section{Conclusion}
  
\end{document}

